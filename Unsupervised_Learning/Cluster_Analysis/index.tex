% Options for packages loaded elsewhere
\PassOptionsToPackage{unicode}{hyperref}
\PassOptionsToPackage{hyphens}{url}
%
\documentclass[
]{article}
\usepackage{amsmath,amssymb}
\usepackage{iftex}
\ifPDFTeX
  \usepackage[T1]{fontenc}
  \usepackage[utf8]{inputenc}
  \usepackage{textcomp} % provide euro and other symbols
\else % if luatex or xetex
  \usepackage{unicode-math} % this also loads fontspec
  \defaultfontfeatures{Scale=MatchLowercase}
  \defaultfontfeatures[\rmfamily]{Ligatures=TeX,Scale=1}
\fi
\usepackage{lmodern}
\ifPDFTeX\else
  % xetex/luatex font selection
\fi
% Use upquote if available, for straight quotes in verbatim environments
\IfFileExists{upquote.sty}{\usepackage{upquote}}{}
\IfFileExists{microtype.sty}{% use microtype if available
  \usepackage[]{microtype}
  \UseMicrotypeSet[protrusion]{basicmath} % disable protrusion for tt fonts
}{}
\makeatletter
\@ifundefined{KOMAClassName}{% if non-KOMA class
  \IfFileExists{parskip.sty}{%
    \usepackage{parskip}
  }{% else
    \setlength{\parindent}{0pt}
    \setlength{\parskip}{6pt plus 2pt minus 1pt}}
}{% if KOMA class
  \KOMAoptions{parskip=half}}
\makeatother
\usepackage{xcolor}
\usepackage[margin=1in]{geometry}
\usepackage{color}
\usepackage{fancyvrb}
\newcommand{\VerbBar}{|}
\newcommand{\VERB}{\Verb[commandchars=\\\{\}]}
\DefineVerbatimEnvironment{Highlighting}{Verbatim}{commandchars=\\\{\}}
% Add ',fontsize=\small' for more characters per line
\usepackage{framed}
\definecolor{shadecolor}{RGB}{248,248,248}
\newenvironment{Shaded}{\begin{snugshade}}{\end{snugshade}}
\newcommand{\AlertTok}[1]{\textcolor[rgb]{0.94,0.16,0.16}{#1}}
\newcommand{\AnnotationTok}[1]{\textcolor[rgb]{0.56,0.35,0.01}{\textbf{\textit{#1}}}}
\newcommand{\AttributeTok}[1]{\textcolor[rgb]{0.13,0.29,0.53}{#1}}
\newcommand{\BaseNTok}[1]{\textcolor[rgb]{0.00,0.00,0.81}{#1}}
\newcommand{\BuiltInTok}[1]{#1}
\newcommand{\CharTok}[1]{\textcolor[rgb]{0.31,0.60,0.02}{#1}}
\newcommand{\CommentTok}[1]{\textcolor[rgb]{0.56,0.35,0.01}{\textit{#1}}}
\newcommand{\CommentVarTok}[1]{\textcolor[rgb]{0.56,0.35,0.01}{\textbf{\textit{#1}}}}
\newcommand{\ConstantTok}[1]{\textcolor[rgb]{0.56,0.35,0.01}{#1}}
\newcommand{\ControlFlowTok}[1]{\textcolor[rgb]{0.13,0.29,0.53}{\textbf{#1}}}
\newcommand{\DataTypeTok}[1]{\textcolor[rgb]{0.13,0.29,0.53}{#1}}
\newcommand{\DecValTok}[1]{\textcolor[rgb]{0.00,0.00,0.81}{#1}}
\newcommand{\DocumentationTok}[1]{\textcolor[rgb]{0.56,0.35,0.01}{\textbf{\textit{#1}}}}
\newcommand{\ErrorTok}[1]{\textcolor[rgb]{0.64,0.00,0.00}{\textbf{#1}}}
\newcommand{\ExtensionTok}[1]{#1}
\newcommand{\FloatTok}[1]{\textcolor[rgb]{0.00,0.00,0.81}{#1}}
\newcommand{\FunctionTok}[1]{\textcolor[rgb]{0.13,0.29,0.53}{\textbf{#1}}}
\newcommand{\ImportTok}[1]{#1}
\newcommand{\InformationTok}[1]{\textcolor[rgb]{0.56,0.35,0.01}{\textbf{\textit{#1}}}}
\newcommand{\KeywordTok}[1]{\textcolor[rgb]{0.13,0.29,0.53}{\textbf{#1}}}
\newcommand{\NormalTok}[1]{#1}
\newcommand{\OperatorTok}[1]{\textcolor[rgb]{0.81,0.36,0.00}{\textbf{#1}}}
\newcommand{\OtherTok}[1]{\textcolor[rgb]{0.56,0.35,0.01}{#1}}
\newcommand{\PreprocessorTok}[1]{\textcolor[rgb]{0.56,0.35,0.01}{\textit{#1}}}
\newcommand{\RegionMarkerTok}[1]{#1}
\newcommand{\SpecialCharTok}[1]{\textcolor[rgb]{0.81,0.36,0.00}{\textbf{#1}}}
\newcommand{\SpecialStringTok}[1]{\textcolor[rgb]{0.31,0.60,0.02}{#1}}
\newcommand{\StringTok}[1]{\textcolor[rgb]{0.31,0.60,0.02}{#1}}
\newcommand{\VariableTok}[1]{\textcolor[rgb]{0.00,0.00,0.00}{#1}}
\newcommand{\VerbatimStringTok}[1]{\textcolor[rgb]{0.31,0.60,0.02}{#1}}
\newcommand{\WarningTok}[1]{\textcolor[rgb]{0.56,0.35,0.01}{\textbf{\textit{#1}}}}
\usepackage{graphicx}
\makeatletter
\def\maxwidth{\ifdim\Gin@nat@width>\linewidth\linewidth\else\Gin@nat@width\fi}
\def\maxheight{\ifdim\Gin@nat@height>\textheight\textheight\else\Gin@nat@height\fi}
\makeatother
% Scale images if necessary, so that they will not overflow the page
% margins by default, and it is still possible to overwrite the defaults
% using explicit options in \includegraphics[width, height, ...]{}
\setkeys{Gin}{width=\maxwidth,height=\maxheight,keepaspectratio}
% Set default figure placement to htbp
\makeatletter
\def\fps@figure{htbp}
\makeatother
\setlength{\emergencystretch}{3em} % prevent overfull lines
\providecommand{\tightlist}{%
  \setlength{\itemsep}{0pt}\setlength{\parskip}{0pt}}
\setcounter{secnumdepth}{5}
\newlength{\cslhangindent}
\setlength{\cslhangindent}{1.5em}
\newlength{\csllabelwidth}
\setlength{\csllabelwidth}{3em}
\newlength{\cslentryspacingunit} % times entry-spacing
\setlength{\cslentryspacingunit}{\parskip}
\newenvironment{CSLReferences}[2] % #1 hanging-ident, #2 entry spacing
 {% don't indent paragraphs
  \setlength{\parindent}{0pt}
  % turn on hanging indent if param 1 is 1
  \ifodd #1
  \let\oldpar\par
  \def\par{\hangindent=\cslhangindent\oldpar}
  \fi
  % set entry spacing
  \setlength{\parskip}{#2\cslentryspacingunit}
 }%
 {}
\usepackage{calc}
\newcommand{\CSLBlock}[1]{#1\hfill\break}
\newcommand{\CSLLeftMargin}[1]{\parbox[t]{\csllabelwidth}{#1}}
\newcommand{\CSLRightInline}[1]{\parbox[t]{\linewidth - \csllabelwidth}{#1}\break}
\newcommand{\CSLIndent}[1]{\hspace{\cslhangindent}#1}
\ifLuaTeX
  \usepackage{selnolig}  % disable illegal ligatures
\fi
\IfFileExists{bookmark.sty}{\usepackage{bookmark}}{\usepackage{hyperref}}
\IfFileExists{xurl.sty}{\usepackage{xurl}}{} % add URL line breaks if available
\urlstyle{same}
\hypersetup{
  pdftitle={Unsupervised Learning: Hierarchical Cluster Analysis},
  pdfauthor={Robert Adongo},
  hidelinks,
  pdfcreator={LaTeX via pandoc}}

\title{Unsupervised Learning: Hierarchical Cluster Analysis}
\author{Robert Adongo}
\date{}

\begin{document}
\maketitle

\hypertarget{introduction}{%
\section{Introduction}\label{introduction}}

\hypertarget{motivation-and-goals}{%
\subsection{Motivation and Goals}\label{motivation-and-goals}}

Clustering is a fundamental technique in unsupervised learning, aimed at
grouping similar observations based on their characteristics. It is
especially useful when no predefined class labels exist and a researcher
seeks to uncover hidden structures in the data
(\protect\hyperlink{ref-kaufman2009finding}{Kaufman \& Rousseeuw,
2009}).

In this project, we implement the \texttt{agnes} function and the
\texttt{diana} function from the \texttt{cluster} package in R to
perform agglomerative and divisive hierarchical clustering,
respectively. Agglomerative methods start with each observation in its
own cluster and iteratively merge the most similar clusters, while
divisive methods begin with all observations in a single cluster and
recursively split them based on dissimilarity
(\protect\hyperlink{ref-kaufman2009finding}{Kaufman \& Rousseeuw, 2009};
\protect\hyperlink{ref-macnaughton1965some}{Macnaughton-Smith, 1965}).

The goal of this section is to provide a theoretical background for
hierarchical clustering, apply both methods to our dataset, and compare
the resulting cluster structures using dendrograms and cluster
evaluation metrics.

\hypertarget{the-agnes-function}{%
\subsection{\texorpdfstring{The \texttt{agnes}
Function}{The agnes Function}}\label{the-agnes-function}}

\hypertarget{the-diana-function}{%
\subsection{\texorpdfstring{The \texttt{diana}
Function}{The diana Function}}\label{the-diana-function}}

\hypertarget{mathematical-setup}{%
\section{Mathematical Setup}\label{mathematical-setup}}

\hypertarget{dissimilarity-and-proximity-matrices}{%
\subsection{Dissimilarity and Proximity
Matrices}\label{dissimilarity-and-proximity-matrices}}

Perfect, Robert. Based on your detailed notes and your request for
\textbf{automatic equation numbering} and consistent notation, here is a
clean R Markdown section written in math-mode-ready LaTeX using
\texttt{bookdown} or \texttt{rmarkdown::pdf\_document} with
\texttt{number\_sections:\ true} and \texttt{mathjax:\ default} or
\texttt{latex\_engine:\ xelatex} to automatically number equations.

We will build the \textbf{Mathematical Setup for Dissimilarity Matrices}
with the same notations as in your earlier explanation (e.g.,
\(d_{ii'}\), \(s_{ii'}\), etc.). Equations will be numbered
automatically using LaTeX
\texttt{\textbackslash{}begin\{equation\}\ ...\ \textbackslash{}end\{equation\}}.

\begin{center}\rule{0.5\linewidth}{0.5pt}\end{center}

\hypertarget{r-markdown-setup-yaml-header}{%
\subsection{✅ R Markdown Setup (YAML
Header)}\label{r-markdown-setup-yaml-header}}

Make sure your R Markdown starts with something like this for equation
numbering:

\begin{Shaded}
\begin{Highlighting}[]
\PreprocessorTok{{-}{-}{-}}
\FunctionTok{title}\KeywordTok{:}\AttributeTok{ }\StringTok{"Clustering: Mathematical Setup"}
\FunctionTok{author}\KeywordTok{:}\AttributeTok{ }\StringTok{"Robert Adongo"}
\FunctionTok{output}\KeywordTok{:}
\AttributeTok{  }\FunctionTok{pdf\_document}\KeywordTok{:}
\AttributeTok{    }\FunctionTok{number\_sections}\KeywordTok{:}\AttributeTok{ }\CharTok{true}
\AttributeTok{    }\FunctionTok{latex\_engine}\KeywordTok{:}\AttributeTok{ xelatex}
\AttributeTok{    }\FunctionTok{keep\_tex}\KeywordTok{:}\AttributeTok{ }\CharTok{true}
\AttributeTok{  }\FunctionTok{html\_document}\KeywordTok{:}
\AttributeTok{    }\FunctionTok{number\_sections}\KeywordTok{:}\AttributeTok{ }\CharTok{true}
\AttributeTok{    }\FunctionTok{mathjax}\KeywordTok{:}\AttributeTok{ default}
\PreprocessorTok{{-}{-}{-}}
\end{Highlighting}
\end{Shaded}

Now here's the content for the \textbf{Mathematical Setup} section:

\begin{center}\rule{0.5\linewidth}{0.5pt}\end{center}

\hypertarget{mathematical-setup-for-dissimilarity-matrices}{%
\subsection{1. Mathematical Setup for Dissimilarity
Matrices}\label{mathematical-setup-for-dissimilarity-matrices}}

\hypertarget{proximity-matrices}{%
\subsubsection{Proximity Matrices}\label{proximity-matrices}}

In clustering, we often represent relationships between objects using a
\textbf{proximity matrix}. Let \(N\) be the number of objects. The
proximity matrix \(D \in \mathbb{R}^{N \times N}\) has entries
\(d_{ii'}\), where:

\begin{equation}
d\_{ii'} \geq 0
\end{equation}

\begin{equation}
d\_{ii} = 0 \quad \text{for all } i = 1, \dots, N
\end{equation}

\begin{equation}
d\_{ii'} = d\_{i'i}
\end{equation}

If the data is collected as \textbf{similarities} \(s_{ii'}\), we
convert it to dissimilarities using a monotone decreasing function, such
as:

\begin{equation}
d\_{ii'} = \frac{1}{s\_{ii'}} \quad \text{or} \quad d\_{ii'} = \max(S) - s\_{ii'}
\end{equation}

If the proximity matrix is not symmetric, we enforce symmetry by
averaging:

\begin{equation}
D \leftarrow \frac{D + D^T}{2}
\end{equation}

Note that subjectively judged dissimilarities may \textbf{not} satisfy
the triangle inequality:

\begin{equation}
d\_{ii'} \leq d\_{ik} + d\_{i'k}, \quad \forall k \in {1, \dots, N}
\end{equation}

\hypertarget{attribute-based-dissimilarities}{%
\subsubsection{Attribute-Based
Dissimilarities}\label{attribute-based-dissimilarities}}

Suppose we have \(p\) attributes (variables), and
\(\mathbf{x}_i = (x_{i1}, x_{i2}, \dots, x_{ip})\) denotes the feature
vector for object \(i\). We define a dissimilarity between each
attribute:

\begin{equation}
d\_j(x\_{ij}, x\_{i'j})
\end{equation}

Then the overall dissimilarity between objects \(i\) and \(i'\) is
computed as:

\begin{equation}
D(\mathbf{x}*i, \mathbf{x}*{i'}) = \sum\_{j=1}^p d\_j(x\_{ij}, x\_{i'j})
\end{equation}

For \textbf{quantitative variables}, the most common choice is
\textbf{squared difference}:

\begin{equation}
d\_j(x\_{ij}, x\_{i'j}) = (x\_{ij} - x\_{i'j})^2
\end{equation}

An alternative approach uses the \textbf{correlation-based similarity}:

\begin{equation}
\rho(\mathbf{x}*i, \mathbf{x}*{i'}) = \frac{\sum\_{j=1}^p (x\_{ij} - \bar{x}*i)(x*{i'j} - \bar{x}*{i'})}{\sqrt{\sum*{j=1}^p (x\_{ij} - \bar{x}*i)^2} \sqrt{\sum*{j=1}^p (x\_{i'j} - \bar{x}\_{i'})^2}}
\end{equation}

where \(\bar{x}_i = \frac{1}{p} \sum_{j=1}^p x_{ij}\). Correlation-based
dissimilarity is then:

\begin{equation}
d(\mathbf{x}*i, \mathbf{x}*{i'}) \propto 2(1 - \rho(\mathbf{x}*i, \mathbf{x}*{i'}))
\end{equation}

\hypertarget{weighted-dissimilarities}{%
\subsubsection{Weighted
Dissimilarities}\label{weighted-dissimilarities}}

We may weigh the importance of each attribute using weights \(w_j\),
with \(\sum_{j=1}^p w_j = 1\). The dissimilarity becomes:

\begin{equation}
D(\mathbf{x}*i, \mathbf{x}*{i'}) = \sum\_{j=1}^p w\_j \cdot d\_j(x\_{ij}, x\_{i'j})
\end{equation}

The average dissimilarity on attribute \(j\) is:

\begin{equation}
\bar{d}*j = \frac{1}{N^2} \sum*{i=1}^N \sum\_{i'=1}^N d\_j(x\_{ij}, x\_{i'j})
\end{equation}

To give equal influence to all attributes, a common (though sometimes
suboptimal) choice is:

\begin{equation}
w\_j = \frac{1}{\bar{d}\_j}
\end{equation}

\hypertarget{special-case-squared-euclidean-distance}{%
\subsubsection{Special Case: Squared Euclidean
Distance}\label{special-case-squared-euclidean-distance}}

If all variables are quantitative and
\(d_j(x_{ij}, x_{i'j}) = (x_{ij} - x_{i'j})^2\), then:

\begin{equation}
D(\mathbf{x}*i, \mathbf{x}*{i'}) = \sum\_{j=1}^p w\_j (x\_{ij} - x\_{i'j})^2
\end{equation}

Further, if \(\mathrm{Var}(X_j)\) is the sample variance of the \(j\)-th
attribute, then:

\begin{equation}
\bar{d}*j = \frac{1}{N^2} \sum*{i=1}^N \sum\_{i'=1}^N (x\_{ij} - x\_{i'j})^2 = 2 \cdot \mathrm{Var}(X\_j)
\end{equation}

\hypertarget{references}{%
\section*{References}\label{references}}
\addcontentsline{toc}{section}{References}

\hypertarget{refs}{}
\begin{CSLReferences}{1}{0}
\leavevmode\vadjust pre{\hypertarget{ref-kaufman2009finding}{}}%
Kaufman, L., \& Rousseeuw, P. J. (2009). \emph{Finding groups in data:
An introduction to cluster analysis}. John Wiley \& Sons.

\leavevmode\vadjust pre{\hypertarget{ref-macnaughton1965some}{}}%
Macnaughton-Smith, P. (1965). \emph{Some statistical and other numerical
techniques for classifying individuals}.

\end{CSLReferences}

\end{document}
